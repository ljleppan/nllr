\documentclass[12pt,a4paper]{article}
\usepackage[utf8]{inputenc}
\usepackage[finnish]{babel}
\usepackage[T1]{fontenc}

\usepackage{amsmath}
\usepackage{amsfonts}
\usepackage{amssymb}
\usepackage[osf,sc]{mathpazo}

\usepackage{float}

\usepackage{color}

\definecolor{mygreen}{rgb}{0,0.6,0}
\definecolor{mygray}{rgb}{0.5,0.5,0.5}
\definecolor{mymauve}{rgb}{0.58,0,0.82}

\usepackage{listings}
\lstset{ %
  backgroundcolor=\color{white},   % choose the background color; you must add \usepackage{color} or \usepackage{xcolor}
  basicstyle=\footnotesize,        % the size of the fonts that are used for the code
  breakatwhitespace=false,         % sets if automatic breaks should only happen at whitespace
  breaklines=true,                 % sets automatic line breaking
  captionpos=b,                    % sets the caption-position to bottom
  commentstyle=\color{mygreen},    % comment style
  extendedchars=true,              % lets you use non-ASCII characters; for 8-bits encodings only, does not work with UTF-8
  frame=single,                    % adds a frame around the code
  keepspaces=true,                 % keeps spaces in text, useful for keeping indentation of code (possibly needs columns=flexible)
  keywordstyle=\color{blue},       % keyword style
  language=Java,                 % the language of the code
  numbers=left,                    % where to put the line-numbers; possible values are (none, left, right)
  numbersep=5pt,                   % how far the line-numbers are from the code
  numberstyle=\tiny\color{mygray}, % the style that is used for the line-numbers
  rulecolor=\color{black},         % if not set, the frame-color may be changed on line-breaks within not-black text (e.g. comments (green here))
  showspaces=false,                % show spaces everywhere adding particular underscores; it overrides 'showstringspaces'
  showstringspaces=false,          % underline spaces within strings only
  showtabs=false,                  % show tabs within strings adding particular underscores
  stepnumber=1,                    % the step between two line-numbers. If it's 1, each line will be numbered
  stringstyle=\color{mymauve},     % string literal style
  tabsize=2,                       % sets default tabsize to 2 spaces
}

\usepackage[a4paper,right=3cm,left=3cm,vmargin=3cm]{geometry}
\linespread{1.6}
\usepackage[osf,sc]{mathpazo}

\tolerance=1
\emergencystretch=\maxdimen
\hyphenpenalty=10000
\hbadness=10000

\clubpenalty=10000
\widowpenalty=10000

\author{Leo Leppänen}
\title{NLLR \\ Toteutusdokumentti}

\begin{document}

\maketitle

\tableofcontents

\pagebreak

\section{Algoritmit}

\subsection{Argmax.single}

\begin{figure}[H]
\begin{lstlisting}
public Result<T> single(Algorithm algorithm, ArrayList<T> args, Object[] constants){
                        
  Object[] argList = new Object[constants.length + 1];
  System.arraycopy(constants, 0, argList, 1, constants.length);      
  argList[0] = args.get(0);
  
  double maxVal = algorithm.calculate(argList);
  T maxArg = args.get(0);
      
  for (int i = 1; i < args.size(); i++) {
    argList[0] = args.get(i);
    double result = algorithm.calculate(argList);
    
    if (maxVal < result){
      maxVal = result;
      maxArg = args.get(i);
    }
  }  
  return new Result(maxArg, maxVal);
}
\end{lstlisting}
\caption{Argmax.single()}
\end{figure}

\subsubsection*{Tilavaativuus}
Tilavaativuus on $\mathcal{O}(1)$, sillä algoritmit käyttää vakiomäärää muuttujia ja tulostaa aina yhden Result-olion.

\subsubsection*{Aikavaativuus}
Argmax suorittaa syötteenään saamansa algoritmin kerran per syötteenä saadun argumenttilistan argumentti, joten aikavaativuus on $\mathcal{O}(A \times n)$, missä $n$ on maksimoitavan argumentin kandidaattien määrä ja $A$ on suoritettavan algoritmin aikavaativuus.


\subsection{Argmax.multiple}

\begin{figure}[H]
\begin{lstlisting}
public ArrayList<Result<T>> multiple(Algorithm algorithm, int amount, ArrayList<T> args, Object[] constants){
  ArrayList<Result<T>> results = new ArrayList<>();
        
  Object[] argList = new Object[constants.length + 1];
  System.arraycopy(constants, 0, argList, 1, constants.length);
  for (T arg : args){
    argList[0] = arg;
    double result = algorithm.calculate(argList);
    if (results.size() < amount){
      results.add(new Result(arg, result));
      sort(results);
    } else if (results.get(amount-1).getValue() < result){
      results.remove(amount-1);
      results.add(new Result(arg, result));
      sort(results);
    }
  }
  return results;
}
\end{lstlisting}
\caption{Argmax.multiple()}
\end{figure}

\begin{figure}[H]
\begin{lstlisting}
private void sort(ArrayList<Result<T>> results){
  for (int i = 1; i < results.size(); i++) {
    Result x = results.get(i);
    int j = i;
    while (j > 0 && results.get(j-1).getValue() < x.getValue()){
      results.set(j, results.get(j-1));
      j = j-1;
    }
    results.set(j, x);
  }
}
\end{lstlisting}
\caption{Argmax.sort()}
\end{figure}

\subsubsection*{Tilavaativuus}
Tilavaativuus on $\mathcal{O}(n)$, sillä kerrallaan muistissa pidetään korkeintaan $n+1$ Result-oliota sekä vakiomäärää muita muuttujia.

\subsubsection*{Aikavaativuus}
Algoritmi suorittaa syötteenä saadun $A$ aikavaativuuksisen algoritmin $n$ kertaa, jolloin tältä osin aikavaativuus on $\mathcal{O}(A \times n)$. Lisäksi pahimmillaan $n$ kertaa kutsutaan metodia sort(), joka järjestää tuloslistan.

Järjestysalgoritmina toimii InsertionSort. Järjestysalgoritmin valintaan vaikutti uniikki konteksti: jokaisella järjestyskerralla kaikki paitsi yksi alkio ovat valmiina oikeilla paikoillaan. Lisäksi järjestettävät taulukot erittäin pienikokoisia. Näissä tapauksissa InsertionSort on nopein ja tehokkain\footnote{http://dl.acm.org/citation.cfm?doid=359024.359026}. Tässä erityistapauksessamme aikavaativuus on lähempänä $\mathcal{O}(n)$ kuin $\mathcal{O}(n^2)$ ja tilavaativuus on $\mathcal{O}(1)$.


\subsection{TFIDF}

\begin{figure}[H]
\begin{lstlisting}
public static double tfidf(String token, Document query, Corpus reference){
  int tf = query.getFrequency(token); 
  double idf = idf(reference, token);
  return tf * idf;
}
\end{lstlisting}
\caption{Tfidf.tfidf()}
\end{figure}

\begin{figure}[H]
\begin{lstlisting}
public static double idf(Corpus reference, String token) {
  int totalDocs = reference.getDocuments().size();
  int docsContainingTerm = reference.numOfDocsContainingToken(token);
        
  return Math.log( (double) totalDocs / docsContainingTerm);
}
\end{lstlisting}
\caption{Tfidf.idf()}
\end{figure}

\subsubsection*{Tilavaativuus}
Algoritmin tilavaativuus on $\mathcal{O}(1)$, sillä se käyttää syötteen koosta riippumatta vain vakiomäärää omia muuttujia.

\subsubsection*{Aikavaativuus}
Algoritmin aikavaativuus voidaan laskea kahdessa osassa: \textit{IDF}-algoritmin osuus sekä \textit{TF-IDF}-algoritmin kokonaisaikavaativuus.

\textit{IDF}-osion aikavaativuus on $\mathcal{O}(1)$. Rivin \textit{reference.getDocuments().size();} suorittaminen vie $\mathcal{O}(1)$, sillä \textit{referece.getDocuments()} palauttaa \textit{ArrayList}-olion ajassa $\mathcal{O}(1)$ ja tältä oliolta voidaan edelleen kysyä ajassa $\mathcal{O}(1)$ sen kokoa. Kutsun \textit{reference.numOfDocsContainingToken(token)} aikavaativuus on myöskin $\mathcal{O}(1)$, sillä kutsussa suoritetaan \textit{HashMap}-olion \textit{containsKey()}- ja \textit{get()}-metodit, jotka molemmat ovat $\mathcal{O}(1)$. Myös lopussa suoritettava \textit{Math.log( (double) totalDocs / docsContainingTerm)} vie aikaa $\mathcal{O}(1)$, jolloin \textit{IDF}-osion aikavaativuus on $\mathcal{O}(1+1+1) = \mathcal{O}(1)$

\textit{TF-IDF} suorittaa yhden \textit{IDF}-kutsun lisäksi yhden \textit{query.getFrequency(token)}-kutsun, joka pahimmassa tapauksessa suorittaa ensin \textit{HashMap.containsKey()}-kutsun ja sen jälkeen \textit{HashMap.get()}-kutsun, joiden molempien aikavaativuus on $\mathcal{O}(1)$. Lisäksi suoritetaan lopuksi vakioaikainen laskutoimitus. Kokonaisuudessaan aikavaativuus on siis $\mathcal{O}(1+1) = \mathcal{O}(1)$


\subsection{NLLR}

\begin{figure}[H]
\begin{lstlisting}
public double calculateNllr(Document query, Corpus candidate){       
  double nllr = 0;
        
  Object[] constants = {query, corpus};
  ArrayList<Result<String>> results = new Argmax().multiple(
    new Tfidf(),
    NUMBER_OF_TOKENS_TO_ANALYZE,
    query.getUniqueTokens().toArrayList(),
    constants);

  for(Result<String> result : results){
    String uniqueToken = result.getArgument();

    double tokenProbQuery = calculateTokenProbability(uniqueToken, query);
    double tokenProbCandidate = calculateTokenProbability(uniqueToken, candidate);
    double tokenProbCorpus = calculateTokenProbability(uniqueToken, corpus);

    nllr += tokenProbQuery * Math.log(tokenProbCandidate / tokenProbCorpus);
  }
        
  return nllr;
}
\end{lstlisting}
\caption{Nllr.nllr()}
\end{figure}

\begin{figure}[H]
\begin{lstlisting}
public double calculateTokenProbability(String token, BagOfWords bagOfWords) {
  double prob =  (double) bagOfWords.getFrequency(token) / bagOfWords.getTotalTokens();
  if (prob == 0){
    return NONZERO;
  } else {
    return prob;
  }
}
\end{lstlisting}
\caption{Nllr.calculateTokenProbability()}
\end{figure}

\subsubsection*{Tilavaativuus}
$\mathcal{O}(1)$, sillä algoritmi käyttää syötteensä lisäksi vain vakiomäärän tilaa bestTokens-taulukon sekä välitulokset tallentavien muuttujien muodossa.
\subsubsection*{Aikavaativuus}
Algoritmi määrittää aluksi Argmax:lla vakiomäärän parhaan TF-IDF arvon saavia sanoja, joille sen jälkeen kullekin suoritetaan useita $\mathcal{O}(1)$ aikavaativuuksisia calculateTokenPropability-komentoja. Täten aikavaativuus on mallia 
$$\mathcal{O}(Argmax.multiple(TFIDF, n) \times a  + a \times (3 \times calculateTokenProbability + 1))$$
missä $a$ on vakio, oletusarvoisesti $10$ ja $n$ on syötteenä saadun dokumentin sanojen määrä. Koska \textit{TF-IDF} aikavaativuus on $\mathcal{O}(1)$ ja $a$ on vakio, voidaan lauseketta sieventää seuraavaan muotoon: $\mathcal{O}(n + calculateTokenProbability)$.

Metodin \textit{calculateTokenPropability} aikavaativuus puolestaan on $\mathcal{O}(1)$, sillä sen suorittamat metodikutsut ovat vakioaikaisia (\textit{HashMap.containsKey()}, \textit{HashMap.get()}. Näiden metodikutsujen lisäksi suoritetaan vain vakioaikaisia laskentakäskyjä.

Täten \textit{NLLR}:n aikavaativuus on $\mathcal{O}(n + calculateTokenProbability)$, missä $n$ on syötteenä saadun dokumentin sanojen määrä. Näin alhainen aikavaativuus on kuitenkin mahdollista vain, koska suurin osa laskennasta suoritetaan jo dokumentteja korpukseen lisättäessä.

\pagebreak
\section{Tietorakenteet}

\subsection{ArrayList}

\subsubsection*{Tilavaativuus}

\textit{ArrayList} käyttää sisäisesti taulukkoa, joka on korkeintaan vakiokertoimen verran sen sisältämien alkioiden lukumäärää $n$ suurempi. Lisäksi käytetään vakiomäärää kirjanpitomuuttujia. Täten pahimman tapauksen tilavaativuus on mallia $\mathcal{O}(x \times n)$, missä $x$ on vakio ja tilavaativuus voidaan siten kirjoittaa muotoon $\mathcal{O}(n)$.

\subsubsection*{Aikavaativuus}

Analysoidaan aikavaativuuden tärkeimpien komentojen osalta erikseen:

\textit{get()}: Aina $\mathcal{O}(1)$, sillä teemme vakiomäärän vakioaikaisia kutsuja.
\begin{figure}[H]
\begin{lstlisting}
public T get(int index){
  if (validIndex(index)){
    return (T) array[index];
  }
  throw new IndexOutOfBoundsException();
}
\end{lstlisting}
\caption{ArrayList.get()}
\end{figure}


\textit{add()}: Pahimmassa tapauksessa $\mathcal{O}(n)$, mikäli joudumme kasvattamaan taustalla olevan taulukon kokoa. Koska taulukon koko kuitenkin kasvaa kahden potensseissa, joudumme tekemään tämän muutoksen vain $n$ lisäyksen välein ja amortisoitu aikavaativuus on $\mathcal{O}(1)$
\begin{figure}[H]
\begin{lstlisting}
public void add(T value) {
  array[size] = value;
  size++;
  modCount++;
  checkCapacity();
}
\end{lstlisting}
\caption{ArrayList.add()}
\end{figure}

\textit{size()}: $\mathcal{O}(1)$, sillä suoritamme vain yhden vakioaikaisen komennon.
\begin{figure}[H]
\begin{lstlisting}
public int size(){
  return this.size;
}
\end{lstlisting}
\caption{ArrayList.size()}
\end{figure}

\textit{indexOf()}: Käymme pahimmassa tapauksessa koko listan suorittaen jokaiselle alkiolle vakiomäärän $\mathcal{O}(1)$ aikavaativuuksisia vertailuja, sekä lopulta suorittaen $\mathcal{O}(1)$ aikaisen palautuskutsun, jolloin aikavaativuus on $\mathcal{O}(n)$. 
\begin{figure}[H]
\begin{lstlisting}
public int indexOf(T value){
  for (int i = 0; i < size; i++) {
    if ((value == null && array[i] == null) || (value != null && value.equals(array[i]))){
      return i;
    }
  }
  return -1;
}
\end{lstlisting}
\caption{ArrayList.indexOf()}
\end{figure}

\textit{contains()}: Kutsumme metodia \textit{indexOf()} jonka aikavaativuus on $\mathcal{O}(n)$, ja suoritamme vakioaikaisen palautuksen. Täten aikavaativuus on yhteensä $\mathcal{O}(n)$.
\begin{figure}[H]
\begin{lstlisting}
public boolean contains(T value){
  return indexOf(value) != -1;
}
\end{lstlisting}
\caption{ArrayList.contains()}
\end{figure}

\textit{isEmpty()}: Suoritamme vakioaikaisen tarkastuksen ja palautamme sen tuloksen. Lopputuloksena aikavaativuus $\mathcal{O}(1)$.
\begin{figure}[H]
\begin{lstlisting}
public boolean isEmpty(){
  return size == 0;
}
\end{lstlisting}
\caption{ArrayList.isempty()}
\end{figure}

\textit{clear()}: Suoritamme vakiomäärän vakioaikaisia operaatiota, joten aikavaativuus on $\mathcal{O}(1)$.
\begin{figure}[H]
\begin{lstlisting}
public void clear(){
  array = new Object[DEFAULT_SIZE];
  limit = DEFAULT_SIZE;
  modCount++;
  size = 0;
}
\end{lstlisting}
\caption{ArrayList.clear()}
\end{figure}

\textit{remove(int index)}: Tarkistamme ensin ajassa $\mathcal{O}(1)$ onko indeksi käytössä. Tämän jälkeen siirrämme annettuun indeksiin ja sitä seuraavan indeksin alkion ja siirrymme seuraavaan indeksiin, jossa toistamme saman operaation. Täten pahimmassa tapauksessa (indeksi = 0) käymme läpi koko listan tehden $\mathcal{O}(1)$ aikavaativuuksisia sijoitusoperaatioita. Lopuksi tarkistamme onko listan kokoon tehtävä muutoksia, joka pahimmassa tapauksessa kestää $\mathcal{O}(n)$. Täten pahimma tapauksen aikavaativuus on $\mathcal{O}(n+n) = \mathcal{O}(n)$. Jälkimmäinen listan koon muutos amortisoituu aikavaativuuteen $\mathcal{O}(1)$, mutta tämä ei vaikuta lopullisen aikavaativuuden $\mathcal{O}$-notaatioon joka on jokatapauksessa $\mathcal{O}(n)$.
\begin{figure}[H]
\begin{lstlisting}
public void remove(int index){
  if (!validIndex(index)){
    throw new IndexOutOfBoundsException();
  }
        
  while(validIndex(index)){
    array[index] = array[index+1];
    index++;
  }
  size--;
  modCount++;
  checkCapacity();
}
\end{lstlisting}
\caption{ArrayList.remove(int index)}
\end{figure}

\textit{remove(T value)}: Pahimmassa tapauksessa suoritamme ensin $\mathcal{O}(n)$ aikaisen \textit{indexOf}-kutsun ja sen jälkeen kutsumme $\mathcal{O}(n)$ aikaista \textit{remove(int index)}-metodia. Täten pahimman tapauksen aikavaativuus on $\mathcal{O}(n+n) = \mathcal(O){n}$.
\begin{figure}[H]
\begin{lstlisting}
public void remove(T value){
  int index = indexOf(value);
  if (index != -1){
    remove(index);
  }
}
\end{lstlisting}
\caption{ArrayList.remove(T value)}
\end{figure}

\textit{set()}: Aikavaativuus on $\mathcal{O}(1)$, sillä suoritamme vain vakiomäärän vakioaikaisia operaatioita.
\begin{figure}[H]
\begin{lstlisting}
public void set(int index, T value){
  if (index <= size && index >= 0){
    array[index] = value;
  }
}
\end{lstlisting}
\caption{ArrayList.set()}
\end{figure}

Taustalla olevan taulukon koon muutokset vievät, kuten yllä esitettyä $\mathcal{O}(n)$, sillä luomme uuden taulukon ajassa $\mathcal{O}(1)$ ja käymme aiemman taulukon läpi ajassa $\mathcal{O}(n)$, suorittaen jokaisella $n$ alkiolle $\mathcal{O}(n)$ aikavaativuuksisen sijoitusoperaation uuteen taulukkoon. Tämän lisäksi suoritetaan vain vakiomäärä vakioaikaisia operaatioita.
\begin{figure}[H]
\begin{lstlisting}
private void changeCapacity(int newLimit){
  Object[] newArray = new Object[newLimit];
  int smallerLimit = Math.min(newLimit, limit);
  System.arraycopy(array, 0, newArray, 0, smallerLimit);
  array = newArray;
  limit = newLimit;
}
\end{lstlisting}
\caption{ArrayList.changeCapacity()}
\end{figure}


\pagebreak
\subsection{HashMap}

\subsubsection*{Tilavaativuus}
\textit{HashMap}:n tilavaativuus on pahimmassa tapauksessa, kaikkien hajautusarvojen osuessa yhteen hajautustaulun ylivuotoketjuun, $\mathcal{O}(taulukon_koko + n)$, missä $taulkon_koko$ on hajautustaulun taustalla olevan taulukon koko, ja $n$ ym. ylivuotoketjussa olevien alkioiden, ja samalla taulukon kaikkien alkioiden määrä. $taulukon_koko$ puolestaan on pahimmassa tapauksessa, $x \times n$, missä $x$ on jokin vakio, meidän tapauksessamme noin $2$. Täten tilavaativuus on pahimmillaan $\mathcal{O}(2n) = \mathcal{O}(n)$. $n$ puolestaan on tarkemmin sanoen muotoa $|K| + |V|$, missä $|K|$ on hajautustaulun avainten tilavaativuus ja $|V|$ hajautustaulun arvojen tilavaativuus. Täten Lopullinen tilavaativuutemme on $\mathcal{O}(|K| + |V|)$.

\subsubsection*{Aikavaativuus}

Analysoimme aikavaativuuden erikseen jokaiselle metodille.

\textit{get()}: Pahimmassa tapauksessa kaikki hajautustaulun alkiot ovat samassa ylivuotoketjussa, jolloin löydämme oikean ylivuotoketjun ajassa $\mathcal{O}(1)$, minkä jälkeen käymme koko ketjun läpi ajassa $\mathcal{O}(n)$. Täten pahimman tapauksen aikavaativuus on $\mathcal{O}(n)$. Amortisoidusti hajautustaulun arvot kuitenkin jakautuvat tasaisesti koko taulukon alueelle, jolloin aikavaativuus on $\mathcal{O}(1+x)$, missä $x$ on keskimääräinen ylivuotoketjun pituus. Tämän ollessa yksi tai hyvin lähellä sitä, on amortisoitu aikavaativuus siten käytännössä $\mathcal{O}(1)$.
\begin{figure}[H]
\begin{lstlisting}
public V get(K key){
  // Determine in which bucket the Entry should be in
  int index = getIndex(key, size);
  Entry entry = this.array[index];
        
  //Go thru that bucket, looking for the key
  while(entry != null) {
    if (entry.getKey().equals(key)){
      return (V) entry.getValue();
    }
    entry = entry.getNext();
  }
        
  //If bucket is empty or no key found in bucket, return null
  return null;
}
\end{lstlisting}
\caption{HashMap.get()}
\end{figure}

\textit{HashMap.put()}: Löydämme oikean ylivuotolistan ajassa $\mathcal{O}(1)$. Pahimmassa tapauksessa kaikki hajautustaulun arvot ovat samassa ylivuotoketjussa, jolloin joudumme käymään läpi koko ylivuotoketjun ajassa $\mathcal{O}(n)$ennen kuin lisäämme arvon ketjun alkuun ajassa $\mathcal{O}(1)$. 

Tämän jälkeen joudumme vielä pahimmassa tapauksessa uudelleenhajauttamaan koko hajautustaulun ajassa $\mathcal{O}(n)$. Täten pahimman tapauksen aikavaativuus on $\mathcal{O}(n+n) = \mathcal{O}(n)$. Keskimäärin joudumme kuitenkin uudelleenhajauttamaan taulun $n$ lisäyksen välein, joten amortisoidusti lisäyksen osuus aikavaativuudesta on $\mathcal{O}(1)$.

Samoin yhdessä ylivuotoketjussa on amortisoidusti 0 (tai 1, kuitenkin käytännössä hyvin lähellä nollaa) alkiota, jolloin lisäyksenkin kesto on amortisoidusti $\mathcal{O}(1)$. Täten koko armotisoitu aikavaativuus on $\mathcal{O}(1)$.
\begin{figure}[H]
\begin{lstlisting}
public void put(K key, V value){   
  //Determine which bucket the key belongs to
  int index = getIndex(key, size);
        
  //If the bucket is empty, add the Entry as a new bucket
  if (this.array[index] == null){
    Entry e = new Entry<>(key, value);
    this.array[index] = e;    
  } else {
    //An existing bucket was found, go thru the bucket looking for the key
    Entry entry = this.array[index];
    while (entry != null){
      if (entry.getKey().equals(key)){
        //Found key, overwrite current value and exit
        entry.setValue(value);
        return;
      }
      entry = entry.getNext();
    }
            
    //Didn't find key in bucket, add a new entry to start of bucket
    Entry newEntry = new Entry<>(key, value);
    newEntry.setNext(array[index]);
    array[index] = newEntry;
  }
  entries++;
  modCount++;
  checkCapacity();
}
\end{lstlisting}
\caption{HashMap.put()}
\end{figure}

\textit{HashMap.remove()}: Löydämme oikean ylivuotolistan ajassa $\mathcal{O}(1)$. Pahimmassa tapauksessa, kaikkien alkioiden ollessa samassa ylivuotolistassa, joudumme jälleen käymään ylivuotolistaa läpi ajassa $\mathcal{O}(n)$. Itse poisto-operaatio on kuitenkin $\mathcal{O}(1)$, sillä suoritamme vakiomäärän vakioaikaisia operaatioita.

Samoin kuin yllä, mahdollinen taustalistan koon muutos on pahimmassa tapauksessa $\mathcal{O}(n)$, mutta amortisoituu aikaa $\mathcal{O}(1)$. Samoin yllä esitetyllä logiikalla ylivuotolistan läpikäynti vie keskimäärin $\mathcal{O}(1)$, jolloin saamme pahimman tapauksen aikavaativuudeksi $\mathcal{O}(n)$, mutta amortisoiduksi aikavaativuudeksi $\mathcal{O}(1)$.
\begin{figure}[H]
\begin{lstlisting}
public void remove(K key){
  //Find the correct bucket
  int index = getIndex(key, size);
  Entry e = array[index];
        
  //If no bucket present, just exit
  if (e == null){
    return;
  }
        
  //Search the bucket for the key
  Entry previous = null;
  while (e != null){
    Entry next = e.getNext();
    if (e.getKey().equals(key)){
      //Found our entry!
      //If entry is head, update array, else update prev's next().
      if (previous == null){
        array[index] = e.getNext();
      } else {
        previous.setNext(next);
      }
                
      entries--;
      modCount++;
      checkCapacity();
      return;
    }
    previous = e;
    e = next;
  }
}
\end{lstlisting}
\caption{HashMap.remove()}
\end{figure}

\textit{HashMap.containsKey()}: Oikea ylivuotoketju selviää ajassa $\mathcal{O}(1)$. Pahimmassa tapauksessa joudumme käymään läpi koko ylivuotoketjun ajassa $\mathcal{O}(n)$ tehden kullekin alkiolle vakiomäärän vakioaikaisia operaatioita ajassa $\mathcal{O}(1)$. Pahimman tapauksen aikavaativuus on siis $\mathcal{O}(n)$. Amortisoidusti metodin aikavaativuus kuitenkin on - kuten yllä - $\mathcal{O}(1)$. 
\begin{figure}[H]
\begin{lstlisting}
public boolean containsKey(K key){
  if (key == null){
    return false;
  }
        
  int index = getIndex(key, size);
  Entry e = array[index];
  while (e != null){
    if (e.getKey().equals(key)){
      return true;
    }
      e = e.getNext();
    }
  return false;
}
\end{lstlisting}
\caption{HashMap.containsKey()}
\end{figure}

\textit{HashMap.getIndex()}: Suoritamme aina vakiomäärän vakioaikaisia kutsuja, joten aikavaativuus on aina $\mathcal{O}(1)$. Metodin toimintalogiikkaa on avattu enemmän JavaDocissa.
\begin{figure}[H]
\begin{lstlisting}
private int getIndex(K key, int size){
  int hash = key.hashCode();
  hash ^= (hash >>> 20) ^ (hash >>> 12) ^ (hash >>> 7) ^ (hash >>> 4); 
        
  int index = hash & (size - 1);
  return index;
}
\end{lstlisting}
\caption{HashMap.getIndex()}
\end{figure}

Emme esitä tässä taustataulukon koon muutosten koodia sen pituuden vuoksi. Käytännössä kuitenkin tarkistamme ajassa $\mathcal{O}(1)$ onko muutokselle tarvetta. Mikäli tarvetta on, luomme ajassa $\mathcal{O}(1)$ uuden taulukon josta tulee uusi taustataulukkomme. Tämän jälkeen suoritamme uudelleenhajautuksen.

Uudelleenhajautuksessa käymme läpi kaikki taulukon $n$ alkiota ajassa $\mathcal{O}(n)$, suorittaen jokaiselle vakiomäärän komentoja joilla alkiolle selvitettään indeksi uudessa taulukossa ajassa $\mathcal{O}(1)$ ja sijoitetaan se oikean indeksin ylivuotoketjun alkuun ajassa $\mathcal{O}(1)$. Täten uudelleenhajautus vie kokonaisuutenaan aikaa $\mathcal{O}(n)$.

\pagebreak
\subsection{HashSet}

\textit{HashSet}-toteutuksemme on käytännössä wrapperi \textit{HashMap}-rakenteen ympärille, joten aika- ja tilavaativuutemme ovat erittäin vastaavat.

\subsubsection*{Tilavaativuus}

Tilavaativuus on sama kuin $n$ alkiota sisältävän \textit{HashMap}-tietorakenteen, kuitenkin siten että $|V|$ on $x \times n$, missä $x$ on vakio ja $n$ on setin sisältämien alkoiden määrä. Tämä johtuu siitä kuinka asetamme kaikkii \textit{HashMap}in avain-arvo -pareihin arvoksi viitteen samaan \textit{DUMMY}-olioon.

Lisäksi käytämme vakiomäärää vakiokokoisia apumuuttujia, jotka eivät vaikuta lopulliseen tilavaativuuteemme, joka on siis $\mathcal{O}(|K| + n)$, missä $|K|$ on setin sisältämä arvojoukko ja $n$ on \textit{DUMMY}-olio ja $k$ kappaletta viitteitä siihen.

\subsubsection*{Aikavaativuus}

Tarkastellaan aikavaativuuksia erikseen jokaiselle metodille.

\textit{HashSet.add()} tarkistaa ensin (keskimäärin) ajassa $\mathcal{O}(1)$ onko arvo jo setissä, ja jos ei ole niin lisää sen edelleen (keskimäärin) ajassa $\mathcal{O}(1)$. Täten keskimääräinen aikavaativuus on $\mathcal{O}(1)$. 

Pahin tapaus on kuitenkin $\mathcal{O}(n + n) = \mathcal{O}(n)$, mikäli kaikki arvot ovat hajautustaulun samassa ylivuotoketjussa.

\begin{figure}[H]
\begin{lstlisting}
public void add(E e){
  if (e != null){
    if (!map.containsKey(e)) {
      map.put(e, DUMMY);
      size++;
      modCount++;
    }
  }
}
\end{lstlisting}
\caption{HashSet.add()}
\end{figure}

\textit{HashSet.remove()} tarkistaa ensin (keskimäärin) ajassa $\mathcal{O}(1)$ onko arvo rakenteessa ja mikäli on, poistaa sen (keskimäärin) ajassa $\mathcal{O}(1)$.

Samoin kuin yllä, pahimmassa tapauksessa taustalla olevan \textit{HashMap}in kaikki arvot ovat samassa ylivuotolistassa ja molemmat operaatiot vievät $\mathcal{O}(n)$ aikaa. Täten pahimman tapauksen aikavaativuus on $\mathcal{O}(n + n) = \mathcal{O}(n)$, mutta amortisoitu aikavaativuus on $\mathcal{O}(1)$.
\begin{figure}[H]
\begin{lstlisting}
public void remove(E e){
  if (e != null){
    if (map.containsKey(e)) {
      map.remove(e);
      size--;
      modCount++;
    }
  }
}
\end{lstlisting}
\caption{HashSet.remove()}
\end{figure}

\textit{HashSet.contains()} on vain wrapperi \textit{HashMap.containsKey()}-metodille ja sen aikavaativuus on sama kuin em. metodilla, eli pahimmassa tapauksessa $ \mathcal{O}(n)$, mutta amortisoidusti $\mathcal{O}(1)$.
\begin{figure}[H]
\begin{lstlisting}
public boolean contains(E e){
  return map.containsKey(e);
}
\end{lstlisting}
\caption{HashSet.contains()}
\end{figure}

\textit{HashSet.size()} suorittaa vakiomäärän vakioaikaisia operaatioita ja on siten aikavaativuudeltaan $\mathcal{O}(1)$.
\begin{figure}[H]
\begin{lstlisting}
public int size(){
  return size;
}
\end{lstlisting}
\caption{HashSet.size()}
\end{figure}

\textit{HashSet.isEmpty()} suorittaa vakiomäärän vakioaikaisia operaatioita ja on siten aikavaativuudeltaan $\mathcal{O}(1)$.
\begin{figure}[H]
\begin{lstlisting}
public boolean isEmpty(){
  return size == 0;
}
\end{lstlisting}
\caption{HashSet.isEmpty()}
\end{figure}

Metodit \textit{HashSet.removeAll()} ja \textit{HashSet.addAll()} suorittavat molemmat $n$ kertaa oman yksittäismetodinsa ja vastaavat siten $n$ kappaletta kyseisiä kutsuja. 
\end{document}