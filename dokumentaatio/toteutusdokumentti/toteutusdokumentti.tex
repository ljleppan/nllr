\documentclass[12pt,a4paper]{article}
\usepackage[utf8]{inputenc}
\usepackage[finnish]{babel}
\usepackage[T1]{fontenc}

\usepackage{amsmath}
\usepackage{amsfonts}
\usepackage{amssymb}

\usepackage[a4paper,right=3cm,left=3cm,vmargin=3cm]{geometry}
\linespread{1.6}
\usepackage[osf,sc]{mathpazo}

\tolerance=1
\emergencystretch=\maxdimen
\hyphenpenalty=10000
\hbadness=10000

\clubpenalty=10000
\widowpenalty=10000

\author{Leo Leppänen}
\title{NLLR \\ Toteutusdokumentti}

\begin{document}

\maketitle

\section{Algoritmit}

\subsection{Argmax.single}
\subsubsection{Tilavaativuus}
Tilavaativuus on $\mathcal{O}(1)$, sillä algoritmit käyttää vakiomäärää muuttujia ja tulostaa aina yhden Result-olion.

\subsubsection{Aikavaativuus}
Argmax suorittaa syötteenään saamansa algoritmin kerran per syötteenä saadun argumenttilistan argumentti, joten aikavaativuus on $\mathcal{O}(A \times n)$, missä $n$ on maksimoitavan argumentin kandidaattien määrä ja $A$ on suoritettavan algoritmin aikavaativuus..

\subsection{Argmax.multiple}
\subsubsection{Tilavaativuus}
Tilavaativuus on $\mathcal{O}(n)$, sillä kerrallaan muistissa pidetään korkeintaan $n+1$ Result-oliota sekä vakiomäärää muita muuttujia.

\subsubsection{Aikavaativuus}
Algoritmi suorittaa syötteenä saadun $A$ aikavaativuuksisen algoritmin $n$ kertaa, jolloin tältä osin aikavaativuus on $\mathcal{O}(A \times n)$. Lisäksi pahimmillaan $n$ kertaa kutsutaan metodia sort(), joka järjestää tuloslistan.

Järjestysalgoritmina toimii InsertionSort. Järjestysalgoritmin valintaan vaikutti uniikki konteksti: jokaisella järjestyskerralla kaikki paitsi yksi alkio ovat valmiina oikeilla paikoillaan. Lisäksi järjestettävät taulukot erittäin pienikokoisia. Näissä tapauksissa InsertionSort on nopein ja tehokkain \footnote{http://dl.acm.org/citation.cfm?doid=359024.359026}. Tässä erityistapauksessamme aikavaativuus on lähempänä $\mathcal{O}(n)$ kuin $\mathcal{O}(n^2)$ ja tilavaativuus on $\mathcal{O}(1)$.

\subsection{NLLR}
\subsubsection{Tilavaativuus}
$\mathcal{O}(1)$, sillä algoritmi käyttää syötteensä lisäksi vain vakiomäärän tilaa bestTokens-taulukon sekä välitulokset tallentavien muuttujien muodossa.
\subsubsection{Aikavaativuus}
Algoritmi määrittää aluksi Argmax:lla vakiomäärän parhaan TF-IDF arvon saavia sanoja, joille sen jälkeen kullekin suoritetaan useita $\mathcal{O}(1)$ aikavaativuuksisia calculateTokenPropability-komentoja. Täten aikavaativuus on sama kuin  
\subsection{TFIDF}
\subsubsection{Tilavaativuus}
\subsubsection{Aikavaativuus}



\section{Tietorakenteet}

\subsection{ArrayList}
\subsubsection{Tilavaativuus}
\subsubsection{Aikavaativuus}

\subsection{HashMap}
\subsubsection{Tilavaativuus}
\subsubsection{Aikavaativuus}

\subsection{HashSet}
\subsubsection{Tilavaativuus}
\subsubsection{Aikavaativuus}

\end{document}